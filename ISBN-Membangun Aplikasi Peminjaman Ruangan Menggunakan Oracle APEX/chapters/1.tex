\section{Pengertian Oracle Application Express}
Oracle Application Express adalah pengembangan host deklaratif untuk mengembangkan dan menggunakan aplikasi web berbasis database. Berkat fitur bawaan seperti tema antarmuka pengguna, kontrol navigasi, form handler, dan laporan yang fleksibel, Oracle Application Express mempercepat proses pengembangan aplikasi.

Mesin Application Express merender aplikasi secara real time dari data yang disimpan dalam tabel database. Saat Anda membuat atau memperluas aplikasi, Oracle Application Express membuat atau mengubah metadata yang disimpan dalam tabel database. Ketika aplikasi dijalankan, mesin Application Express kemudian membaca metadata dan menampilkan aplikasi.

Untuk memberikan provide stateful dalam suatu aplikasi, Oracle Application Express secara transparan mengelola status sesi dalam database. Pengembang aplikasi dapat memperoleh dan mengatur status sesi menggunakan pergantian sederhana dan sintaks variabel yang terikat dengan SQL standar.

\section{Bagaimana Oracle APEX Dapat Bekerja}
Oracle Application Express diinstal dengan database Oracle Anda dan terdiri dari data dalam tabel, dan kode PL / SQL. Apakah Anda menjalankan pengembangan Oracle Application Express atau menjalankan aplikasi yang dibangun menggunakan Oracle Application Express, maka prosesnya akan sama. Browser Anda mengirimkan permintaan URL yang diterjemahkan ke dalam Oracle Application Express PL / SQL call yang sesuai. Setelah database memproses PL / SQL, hasilnya akan dikirim kembali ke browser Anda sebagai HTML. Siklus ini terjadi setiap kali Anda request atau submit sebuah halaman.

\par Oracle Application Express tidak menggunakan koneksi database khusus. Alih-alih, setiap permintaan dilakukan melalui sesi basis data baru, menggunakan sumber daya CPU minimal. Status sesi aplikasi dikelola dalam tabel database oleh mesin Application Express.

\par Di belakang layar, mesin Application Express merender dan memproses halaman. Mesin Application Express juga dapat melakukan tugas-tugas ini:
\begin{itemize}
    \item Session state management
    \item Layanan otentikasi / Authentication services
    \item Layanan otorisasi / Authorization services
    \item Kontrol aliran halaman / Page flow control
    \item Pemrosesan validasi / Validation processing
\end{itemize}

\section{Latar Belakang}
Meningkatnya jumlah Mahasiswa dalam Peminjaman Ruangan, penelitian ini dibuat untuk menyusun Aplikasi Peminjaman Ruangan mengunakan \textit{Oracle Apex} sehingga dapat dikontrol oleh Staff BAAK dengan cepat  dan mudah dalam mengontrol data. Sistem yang digunakan adalah database Oracle SQL sebagai tempat penyimpanan data dan menggunakan Oracle Apex untuk menampilkan informasi ruangan. Oracle Apex akan menampilkan status pada ruangan tersebut guna mempermudah Mahasiswa untuk mengetahui ruangan yang kosong.

\par \textit{Oracle Aplication Express} adalah suatu website yang dikembangkan oleh Oracle dalam mempermudah mengembangkan aplikasi dengan cepat, kata Express yaitu berarti cepat, dalam percobaan untuk memasukkan sebuah data Excell ke Oracle Apex dengan format excle dapat dengan mudah diproses otomatis oleh sistem Oracle Apex.

\par Dengan perkembangan teknologi informasi yang terbarukan menjadi harapan dunia untuk menggunakan teknologi tersebut. Perkembangan teknologi informasi Indonesia sendiri yang begitu pesat tentu membuka sebuah peluang untuk meningkatkan peluang bisnis pada perusahaan kecil maupun ternama. Banyak perusahaan di Indonesia telah mengubah sistem pemasarannya yang bersifat konvensional menjadi system informasi guna meningkatkan kualitas perusahaan dan daya saing dengan pelaku usaha lainnya demi menunjang keberlangsungan hidup perusahaan dan bisnisnya.

\par Banyak kalangan mulai dari perusahaan besar hingga lembaga-lembaga akademik yang memanfaatkan kemajuan teknologi informasi dan secara langsung dan membawa keuntungan tersendiri bagi yang menggunakannya.

\par Masyarakat akan semakin memilih layanan teknologi dengan melihat keuntungan dari layanan tersebut dan mengikuti perkembangan zaman, Teknologi ibarat sebuah perlombaan bila teknologi tersebut mampu hadir secara cepat akan langsung diburu oleh pelanggan, tetapi jika teknologi tersebut masih tertinggal zaman maka pelanggan banyak meninggalkan teknologi tersebut dan mencari teknologi yang lebih cepat dan efisien, maka itulah yang akan cepat diserap dan diterima oleh konsumen,dalam hal ini merekalah yang menentukansebuah keberhasilan perusahaan dalam meningkatkan layanan dengan menggunakan suatu kemajuan teknologi. 

\par Dalam dunia digital terutama gadget pastinya akan sering memakai aplikasi setiap aplikasi yang dirasa memenuhi kebutuhan semua masnusia 

\par Dalam menggunakan apliaksi dapat mempermudah menginput data Mahasiswa yang akan meminjam ruangan yang kosong, dan dapat mempermudah karyawan BAAK(Bagian Administrasi Akademik) dalam memberi akses untuk Peminjaman Ruangan

\par Begitu juga dalam aplikasi Peminjaman Ruang ini kami akan menganalisa setiap detil dari pertama menjalankannya sampai selesai lalu menganalisa kode yang ada di aplikasi tersebut dan menganalisa database yang digunakan di dalam aplikasi tersubut.

\par Selain itu bagian BAAK ini juga sering melayani Dosen, Himpunan, dan Unit Kegiatan Mahasiswa (UKM) ketika ingin melakukan kegiatan non-akademik dan Akademik untuk Peminjaman Ruangan, bagian inilah yang melakukan penyetujuan peminjaman ruangan ketika unit-unit yang ada di kampus ingin mengadakan kegiatan non-Akademik dan Akademik.

\par Dengan menggunakan Oracle Aplication Express atau disebut dengan Oracle Apex, kami dapat mempermudah membangun aplikasi dengan cepat dan akurat, namun dalam ke-akuratan masih rentang pada 50 sampai 70 persen dikarenakan saat memasukkan data melalui aplikasi excell terkadang harus mengatur sendiri tipe data yang harus digunakan.

\par Dalam buku ini kita akan menjelaskan bagaimana cara membuat Aplikasi yang dibuat dari Oracle APEX Online, selamat bekerja teman-teman.





