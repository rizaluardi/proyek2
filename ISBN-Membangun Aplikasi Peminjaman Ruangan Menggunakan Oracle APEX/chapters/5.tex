\section{Pengertian Trigger}
Trigger adalah blok PL/SQL yang disimpan dalam database dan akan diaktivasi ketika kita melakukan statement-statement SQL (DELETE, UPDATE, dan INSERT) pada sebuah tabel. Aktivasi trigger didasarkan pada event yang terjadi di dalam tabel tersebut sehingga trigger dapat membantu dalam menjaga integritas dan konsistensi data.
\par Implementasi trigger yang sering ditemui dalam dunia nyata adalah untuk mengeset dan mengubah nilai kolom dalam suatu tabel sehingga validasi nilai dari tabel tersebut akan terjaga. Adanya trigger dalam database akan meringankan kita dalam pembuatan aplikasi karena di dalam aplikasi yang kita buat, kita tidak perlu lagi untuk melakukan validasi data.
\par Dalam trigger Oracle telah menyediakan statement CREATE TRIGGER untuk membuat
sebuah trigger yang selanjutnya akan diaktivasi berdasarkan event tertentu. Secara
umum, event trigger terbagi menjadi dua, yaitu BEFORE (sebelum) dan AFTER
(setelah). Event tersebut menandakan kapan trigger akan diaktivasi, apakah sebelum
ataukah sesudah proses yang dilakukan di dalam tabel bersangkutan\cite{mudafiq2003trigger}.
\begin{table}[]
    \begin{center}
\begin{tabular}{ |c|c| } 
 \hline
 Nama Event & Keterangan  \\ \hline 
 BEFORE INSERT & Diaktifkan sekali sebelum Statement INSERT  \\ \hline
 AFTER INSERT & Diaktifkan sekali setelah Statement INSERT  \\ \hline
 BEFORE UPDATE & Diaktifkan sekali sebelum Statement UPDATE \\ \hline
 AFTER UPDATE & Diaktifkan sekali setelah Statement UPDATE \\ \hline
 BEFORE DELETE & Diaktifkan sekali sebelum Statement DELETE \\ \hline
 AFTER DELETE & Diaktifkan sekali setelah Statement DELETE \\
 \hline
\end{tabular}
\caption{Tabel Pengertian Trigger}
    \label{tabel_kebutuhan_sistem}
\end{center}
\end{table}
\section{Pengertian Sequence}
Sequence iaalah salah satu object yang ada di aplikasi Oracle Apex pada database yang digunakan untuk melakukan penomoran otomatis. Namun pada database MySQL yang biasa lebih dikenal dengan nama Auto Increment, Sequence biasanya digunakan sebagai Primary Key untuk membuat ID yang secara otomatis terinput oleh database.
\par pada Oracle Apex database, anda dapat membuat sequence hingga mencapai kelipatan yang tak terbatas, namun anda juga bisa mengaturnya dengan sesuai kebutuhan anda.

\section{Membuat Query Sequence dan Trigger}
Dalam tahap ini yaitu kita akan membuat Query pada SQL Command, tuliskan kode kode berikut

\begin{itemize}
    \item[1] Buat Query Untuk Sequence terlebih dahulu yang nanti akan di integrasikan pada Trigger JABATAN\_TRG
    \begin{lstlisting}
CREATE SEQUENCE   "JABATAN_SEQ"  
 MINVALUE 1 
 MAXVALUE 9999999999999999999999999999 
 NOCYCLE
 CACHE 20
 NOORDER;
    \end{lstlisting}
    \item[2] Buat Query TRIGGER yang nanti akan terhubung pada tabel JABATAN
    \begin{lstlisting}
/*QUERY PERTAMA*/

CREATE OR REPLACE EDITIONABLE TRIGGER  "JABATAN_TRG" 
BEFORE INSERT ON JABATAN_ORMAWA 
FOR EACH ROW

BEGIN
  SELECT JABATAN_SEQ.NEXTVAL
  INTO   :new.KODE_JABATAN
  FROM   dual;
END;

/

/*QUERY KEDUA*/

ALTER TRIGGER  "JABATAN_TRG" ENABLE
/
    \end{lstlisting}
 \end{itemize}
